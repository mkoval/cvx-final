\documentclass[onlymath]{beamer}
\usefonttheme{serif}
\usetheme{default}
\title{
    Minimum Snap Trajectory Generation and Control for Quadrotors
}
\author{
    Michael Koval \\
    mkoval@eden.rutgers.edu \\
    Electrical \& Computer Engineering \\
    Rutgers University
}
\date{April 26, 2012}
\begin{document}

\begin{frame}
\titlepage
\end{frame}

\begin{frame}{State Space}
\begin{itemize}
\item Pose of the robot is $\langle x, y, z, \phi, \theta, \psi \rangle \in
      \mathbb{R}^6$
    \begin{itemize}
    \item $\phi$, $\theta$, and $\psi$ are Z - X - Y Euler angles
    \end{itemize}
\item State is pose along with its first derivatives:
    \begin{itemize}
    \item $\vec{x} = \langle x, y, z, \phi, \theta, \psi, \dot{x}, \dot{y},
          \dot{z}, p , q, r \rangle \in \mathbb{R}^{12}$
    \item $p$, $q$, and $r$ are the components of angular velocity
    \end{itemize}
\item Control $\vec{\omega} \in \mathbb{R}^4$, the four rotor speeds
    \begin{itemize}
    \item Changes net force, $F$, and the body moments, $M_i$
    \item System is governed by $\vec{u} = \langle F, M_x, M_y, M_z \rangle
          = Q \vec{\omega}^2$
    \end{itemize}
\end{itemize}
\begin{center}
    \textbf{This system is underactuated.} \\
    \small
    (There are 12 state variables and only four control variables)
\end{center}
\end{frame}

\begin{frame}{Control Algorithm}
\begin{itemize}
\item Control only four outputs, $\sigma = \langle x, y, z, \psi \rangle$
    \begin{itemize}
    \item State and controls written in terms of $\sigma$, $\dot{\sigma}$, and
          $\ddot{\sigma}$
    \item Planning will done in this smaller subspace
    \end{itemize}
\item Possible to derive a non-linear controller for $\sigma$
    \begin{itemize}
    \item See ?? and ?? for details about the controller
    \item ...but we're only interested in trajectory planning
    \end{itemize}
\item Keyframe ($\sigma_i$): specifies the pose at time $t_i$
    \begin{itemize}
    \item ``Snapshop'' of position and yaw at a given time
    \item Valid trajectory must pass through all keyframes
    \end{itemize}
\end{itemize}
\begin{center}
    \textbf{How do we generate a smooth trajectory that passes through a
    sequence of keyframes?}
\end{center}
\end{frame}

\begin{frame}{Trajectory Generation}
\begin{itemize}
\item Interpolate using $n$-th degree polynomials
\begin{equation*}
    \sigma(t) = \begin{cases}
        \sigma_1(t) = \sum_{i = 0}^n \sigma_{i1} t^i & t_0 \le t < t_1 \\
        \sigma_2(t) = \sum_{i = 0}^n \sigma_{i2} t^i & t_1 \le t < t_2 \\
        \vdots & \\
        \sigma_m(t) = \sum_{i = 0}^n \sigma_{im} t^i & t_{m-1} \le t \le t_m \\
    \end{cases}
\end{equation*}

\item Trajectory must satisfy these requirements:
    \begin{enumerate}
    \item Is as ``smooth'' as possible $\rightarrow$ \textbf{objective}
    \item Pass through all keyframes $\rightarrow$ \textbf{constraint}
    \item Continuity of state and derivatives $\rightarrow$ \textbf{constraint}
    \item Stays within a safe corridor $\rightarrow$ \textbf{constraint}
    \end{enumerate}
\end{itemize}
\end{frame}

\begin{frame}{Non-Convex Optimization Problem}
    \begin{align*}
    & \text{minimize}
        & & \int_{t_0}^{t_m}{ \left[
            \mu_r    \left|\left| \frac{d^{k_r} \vec{r}}{d t^{k_r}} \right|\right|^2
          + \mu_\psi \left( \frac{d^{k_\psi} \psi}{d t^{k_\psi}} \right)^2
        \right] dt} \\
    & \text{subject to}
        & & \sigma_T(t_i) = \hat{\sigma}_i,
        & & i = 1, ..., m \\
    & & & \vec{r}_i(t_i) = \vec{r}_{i + 1}(t_i),
        & & i = 1, ..., m - 1 \\
    & & & \psi_i(t_i) = \psi_{i + 1}(t_i),
        & & i = 1, ..., m - 1 \\
    & & & \nabla_t^j \vec{r_i}(t_i) = \nabla_t^j \vec{r}_{i + 1}(t_i),
      & & i = 1, ..., m - 1, \, j = 1, ..., k_r \\
    & & & \frac{d^j \psi_i}{d t^j} \big|_{t = t_i} = \frac{d^j \psi_{i + 1}}{d t^j} \big|_{t = t_i},
      & & i = 1, ..., m - 1, \, j = 1, ..., k_\psi \\
    & & & ||\vec{d}_i(t)||_\infty \le \delta_i,
      & & i = 1, ..., m - 1, \, t_i \le t < t_{i + 1}
    \end{align*}
    \textbf{Note:} $\vec{d}_i(t)$ the distance from $\vec{r}(t)$ to the line from
          $\hat{\sigma}_{i - 1}$ to $\hat{\sigma}_i$
\end{frame}

\begin{frame}{Rewritten as a Quadratic Program}
    \begin{align*}
    & \text{minimize}
        & & c^T H c + f^T c \\
    & \text{subject to}
        & & \sigma_T(t_i) = \hat{\sigma}_i,
        & & i = 1, ..., m \\
    & & & \vec{r}_i(t_i) = \vec{r}_{i + 1}(t_i),
        & & i = 1, ..., m - 1 \\
    & & & \psi_i(t_i) = \psi_{i + 1}(t_i),
        & & i = 1, ..., m - 1 \\
    & & & \vec{x} \cdot \vec{d}\left(t_i + \frac{j}{1 + n_c}\left(t_{i + 1} - t_i\right)\right)
          \le \delta_i
        & & i = 1, ..., m - 1, j = 1, ..., n_c \\
    & & & \vec{x} \cdot \vec{d}\left(t_i + \frac{j}{1 + n_c}\left(t_{i + 1} - t_i\right)\right)
          \ge -\delta_i
        & & i = 1, ..., m - 1, j = 1, ..., n_c \\
    & & & Ac \preceq b
    \end{align*}
    where
    \begin{align*}
        c   &= \langle x_{ij}, y_{ij}, z_{ij}, \psi_{ij} \rangle \in \mathbb{R}^{4nm} \\
        n_c &= \text{number of segments used to discretize paths between nodes}
        H   &= \text{discrete approximation of the 
    \end{align*}
\end{frame}

\end{document}
