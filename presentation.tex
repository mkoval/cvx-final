\documentclass[onlymath]{beamer}
\usefonttheme{serif}
\usetheme{default}
\title{
    Minimum Snap Trajectory Generation and Control for Quadrotors \\
    \& \\
    Reconfiguration Algorithms for Mobile Robotic Networks
}
\author{
    Michael Koval \\
    mkoval@eden.rutgers.edu \\
    Electrical \& Computer Engineering \\
    Rutgers University
}
\date{April 26, 2012}
\begin{document}

\begin{frame}
\titlepage
\end{frame}

\begin{frame}{Problem Statement}
    \begin{itemize}
    \item Pose of the robot is $\langle x, y, z, \phi, \theta, \psi \rangle \in
          \mathbb{R}^6$
        \begin{itemize}
        \item $\phi$, $\theta$, and $\psi$ are Z - X - Y Euler angles
        \end{itemize}
    \item State is pose along with its first derivatives:
        \begin{itemize}
        \item $\vec{x} = \langle x, y, z, \phi, \theta, \psi, \dot{x}, \dot{y},
              \dot{z}, p , q, r \rangle \in \mathbb{R}^{12}$
        \item $p$, $q$, and $r$ are the components of angular velocity
        \end{itemize}
    \item Pretend the quadrotor's controls are $\sigma = \langle x, y, z, \psi \rangle$
        \begin{itemize}
        \item There exists a controller that maps $\sigma$, $\dot{\sigma}$, and
              $\ddot{\sigma}$ to rotor speeds
        \item Planning will done in this smaller subspace
        \end{itemize}
    \item \textbf{Trajectory}: Series of controls, $\sigma_T(t)$, executed over time
    \item \textbf{Keyframe} ($\hat{\sigma}_i$): desired pose at time $t_i$;
          i.e. $\sigma_T(t_i) = \hat{\sigma}_i$
        \begin{itemize}
        \item Three-dimensional ``waypoints'' that include yaw
        \item Behavior between keyframes is intentionally left unspecified
        \end{itemize}
    \end{itemize}
\begin{center}
    \textbf{How do we generate a ``good'' trajectory that passes through an
            arbitrary sequence of keyframes?}
\end{center}
\end{frame}

\begin{frame}{Trajectory Generation}
    \begin{itemize}
    \item Interpolate between keyframes using $n$-th degree splines:
    \begin{equation*}
        \sigma_T(t) = \begin{cases}
            \sigma_{T1}(t) = \sum_{i = 0}^n \sigma_{Ti1} t^i & :\; t_0 \le t < t_1 \\
            \sigma_{T2}(t) = \sum_{i = 0}^n \sigma_{Ti2} t^i & :\; t_1 \le t < t_2 \\
            \vdots & \\
            \sigma_{Tm}(t) = \sum_{i = 0}^n \sigma_{Tim} t^i & :\; t_{m-1} \le t \le t_m
        \end{cases}
    \end{equation*}
        \begin{itemize}
        \item $\sigma_T(t)$ is parameterized by $\sigma_{ij}$ for $i = 0, ..., n$ and $j = 1, ..., m$
        \item Enforce continuity of derivatives at the knots
        \end{itemize}
    \item Constrain the path to a ``safe corridor'' between keyframes
        \begin{itemize}
        \item Let $\vec{r}_i(t)$ be the $x$, $y$, and $z$ components of $\sigma_{Ti}$
        \item $\vec{d}_i(t) = \left(\vec{r}_T(t) - \vec{r}_i\right) - \left[\left(\vec{r}_T(t) - \vec{r}_i\right) \cdot \vec{t}_i\right] \vec{t}_i$ is the offset
        \item Constrain $\left|\left| \vec{d}_i(t) \right|\right|_\infty \le \delta_i$ where $\delta_i$ is the corridor width
        \end{itemize}
    \end{itemize}
\end{frame}

\begin{frame}{Non-Convex Optimization Problem}
    \begin{align*}
    & \underset{\sigma_{ij}}{\text{minimize}}
        & & \int_{t_0}^{t_m}{ \left[
            \mu_r    \left|\left| \frac{d^{k_r} \vec{r}}{d t^{k_r}} \right|\right|^2
          + \mu_\psi \left( \frac{d^{k_\psi} \psi}{d t^{k_\psi}} \right)^2
        \right] dt} \\
    & \text{subject to}
        & & \sigma_T(t_i) = \hat{\sigma}_i,
        & & i = 0, ..., m \\
    & & & \vec{r}_{i - 1}(t_i) = \vec{r}_i(t_i),
        & & i = 1, ..., m \\
    & & & \psi_{i - 1}(t_i) = \psi_i(t_i),
        & & i = 1, ..., m \\
    & & & \nabla_t^j \vec{r}_{i - 1}(t_i) = \nabla_t^j \vec{r}_i(t_i),
      & & i = 1, ..., m, \, j = 1, ..., k_r \\
    & & & \frac{d^j \psi_{i - 1}}{d t^j} \big|_{t = t_i} = \frac{d^j \psi_i}{d t^j} \big|_{t = t_i},
      & & i = 1, ..., m, \, j = 1, ..., k_\psi \\
    & & & ||\vec{d}_i(t)||_\infty \le \delta_i,
      & & i = 1, ..., m, \, t_i \le t < t_{i + 1}
    \end{align*}
\end{frame}

\begin{frame}{Rewritten as a Quadratic Program}
    \begin{align*}
    & \underset{\sigma_{ij}}{\text{minimize}}
        & & c^T H c + f^T c \\
    & \text{subject to}
        & & \sigma_T(t_i) = \hat{\sigma}_i,
        & & i = 1, ..., m \\
    & & & \vec{r}_i(t_i) = \vec{r}_{i + 1}(t_i),
        & & i = 1, ..., m - 1 \\
    & & & \psi_i(t_i) = \psi_{i + 1}(t_i),
        & & i = 1, ..., m - 1 \\
    & & & \vec{x} \cdot \vec{d}\left(t_i + \frac{j}{1 + n_c}\left(t_{i + 1} - t_i\right)\right)
          \le \delta_i
        & & i = 1, ..., m - 1, j = 1, ..., n_c \\
    & & & \vec{x} \cdot \vec{d}\left(t_i + \frac{j}{1 + n_c}\left(t_{i + 1} - t_i\right)\right)
          \ge -\delta_i
        & & i = 1, ..., m - 1, j = 1, ..., n_c \\
    & \text{where} \\
    & & c   &= \langle x_{ij}, y_{ij}, z_{ij}, \psi_{ij} \rangle \in \mathbb{R}^{4nm} \\
    & & H   &= \text{discrete approximation of $k_r$-th order derivative} \\
    & & f   &= \text{discrete approximation of $k_\psi$-th order derivative} \\
    & & n_c &= \text{number of samples per path segment} \\
    \end{align*}
\end{frame}

\begin{frame}{Summary}
    Summary:
    \begin{enumerate}
    \item Define a series of keypoints and corresponding corridor widths
    \item Solve the optimization problem for the parameters of $\sigma_T(t)$
    \item Use a controller to follow $\sigma_T(t)$ by controlling the rotor speeds
    \end{enumerate}
    \vspace{1em}
    Possible to add additional constraints to the optimization problem:
    \begin{itemize}
    \item initial conditions (e.g. starting velocity) $\Rightarrow$ constraint
    \item final conditions (e.g. ending velocity) $\Rightarrow$ constraint
    \item intermediate conditions $\Rightarrow$ constraint
    \end{itemize}
    \vspace{1em}
    Also...it is possible to find the optimal relative segment lengths by
    solving a separate convex optimization problem!
\end{frame}

\begin{frame}
\end{frame}

\begin{frame}{Problem Statement}
    \begin{itemize}
    \item 
    \end{itemize}
\end{frame}

\end{document}
